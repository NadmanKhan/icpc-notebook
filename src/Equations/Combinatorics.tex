\begin{enumerate}

            \end{enumerate}
\subsubsection{General}\begin{enumerate}

            \item $\displaystyle \displaystyle \sum \limits_{0\leq k \leq n} {n-k \choose k} = Fib_{n+1}$
            \item $\displaystyle \displaystyle {n \choose k}={n \choose n-k}$
            \item $\displaystyle \displaystyle {n \choose k}+{n \choose k+1}={n+1 \choose k+1}$
            \item $\displaystyle \displaystyle k{n \choose k}=n{n-1 \choose k-1}$
            \item $\displaystyle \displaystyle {n \choose k}=\frac{n}{k}{n-1 \choose k-1}$
            \item $\displaystyle \sum \limits_{i=0}^n{n \choose i}=2^n$
            \item $\displaystyle \sum \limits_{i\geq 0}{n \choose 2i}=2^{n-1}$
            \item $\displaystyle \sum \limits_{i\geq 0}{n \choose 2i+1}=2^{n-1}$
            \item $\displaystyle \sum \limits_{i= 0}^k \left( -1 \right) ^i{n \choose i}=\left( -1 \right) ^k{n-1 \choose
                k}$
            
            \item $\displaystyle \sum \limits_{i= 0}^k{n+i \choose i}= \sum \limits_{i= 0}^k{n+i \choose n} = {n+k+1
                \choose
                k}$
            
            \item $\displaystyle \displaystyle1{n \choose 1}+2{n \choose 2}+3{n \choose 3}+…+n{n \choose n}=n2^{n-1}$
            \item $\displaystyle \displaystyle1^2{n \choose 1}+2^2{n \choose 2}+3^2{n \choose 3}+…+n^2{n \choose
                n}=(n+n^2)2^{n-2}$
            \item \textbf{Vandermonde’s Identify:} $\displaystyle \sum \limits_{k=0}^r{m \choose k}{n \choose
                r-k}={m+n
                \choose r}$
            \item \textbf{Hockey-Stick Identify:} $\displaystyle \ n,r \in N, n > r, \sum \limits_{i=r}^n{i
                \choose
                r}={n+1 \choose r+1}$
            \item $\displaystyle \sum \limits_{i=0}^k{k \choose i}^2={2k \choose k}$
            \item $\displaystyle \sum \limits_{k=0}^n{n \choose k}{n \choose n-k}={2n \choose n}$
            \item $\displaystyle \sum \limits_{k=q}^n{n \choose k}{k \choose q}=2^{n-q}{n \choose q}$
            \item $\displaystyle \sum \limits_{i=0}^nk^i{n \choose i}=(k+1)^n$
            \item $\displaystyle \sum \limits_{i=0}^n{2n \choose i}=2^{2n-1}+\frac{1}{2}{2n \choose n}$
            \item $\displaystyle \sum \limits_{i=1}^n{n \choose i}{n-1 \choose i-1}={2n-1 \choose n-1}$
            \item $\displaystyle \sum \limits_{i=0}^n{2n \choose i}^2=\frac{1}{2} \left( {4n \choose 2n}+{2n \choose n}^2
                \right)$
            
            \item \textbf{Highest Power of $\displaystyle 2$ that divides $^{2n}C_n$:}
                Let $\displaystyle x$ be the number of $\displaystyle 1$s in the binary representation. Then the number
                of
                odd
                terms will be $\displaystyle 2^x$.Let it form a sequence. The $n$-th value in the sequence (starting
                from
                $n$ =
                0) gives the highest power of $\displaystyle 2$ that divides $^{2n}C_n$.
            \item \textbf{Pascal Triangle}
                \begin{enumerate}

                    \item In a row $\displaystyle p$ where $\displaystyle p$ is a prime number, all the terms in that row
                        except
                        the $\displaystyle 1$s are multiples of $\displaystyle p$.
                    \item Parity: To count odd terms in row $n$, convert $n$ to binary. Let $\displaystyle x$ be the
                        number of
                        $\displaystyle 1$s in the binary representation. Then the number of odd terms will be
                        $\displaystyle
                        2^x$.
                    \item Every entry in row $\displaystyle 2^n-1, n\geq 0,$ is odd.
                \end{enumerate}

            
            \item An integer $n\geq 2$ is prime if and only if all the intermediate binomial coefficients $\displaystyle
                \displaystyle {n \choose 1},{n \choose 2},…, {n \choose n-1}$ are divisible by $n$.
            \item \textbf{Kummer’s Theorem:} For given integers $n\geq m\geq 0$ and a prime number $\displaystyle
                p$,
                the
                largest power of $\displaystyle p$ dividing $\displaystyle \displaystyle {n \choose m}$ is equal to the
                number
                of carries when $m$ is added to $n$-$m$ in base $\displaystyle p$. For implementation take inspiration
                from
                lucas theorem.
            \item Number of different binary sequences of length $n$ such that no two $0$’s are adjacent=$\displaystyle
                Fib_{n+1}$
            
            \item \textbf{Combination with repetition:} Let’s say we choose $\displaystyle \displaystyle k$
                elements
                from
                an $n$-element set, the order doesn’t matter and each element can be chosen more than once. In that
                case,
                the
                number of different combinations is: $\displaystyle \displaystyle {n+k-1 \choose k}$
            \item 
                Number of ways to divide $n$ persons in $\displaystyle \frac{n}{k}$ equal groups i.e. each having
                    size
                    $\displaystyle \displaystyle k$ is

                \[\displaystyle \frac{n!}{k!^{\frac{n}{k}} \left( \frac{n}{k} \right) !}= \prod \limits_{n \geq
                k}^{n-=k}{n-1
                \choose k-1}\]
            
            \item The number non-negative solution of the equation: $\displaystyle x_1+x_2+x_3+…+x_k=n \text{ is}{n+k-1
                \choose
                n}$
            \item Number of ways to choose $n$ ids from $\displaystyle 1$ to b such that every id has distance at least k
                $\displaystyle \displaystyle =\left( \frac{b-(n-1)(k-1)}{n} \right)$
            \item $\displaystyle \displaystyle \sum \limits_{i=1,3,5,\dots}^{i\leq n} \binom{n}{i} a^{n-i}b^i =
                \frac{1}{2}
                ((a+b)^n-(a-b)^n)$
            \item $\displaystyle \displaystyle \sum \limits_{i=0}^{n} \dfrac{\binom{k}{i}}{\binom{n}{i}} =
                \dfrac{\binom{n+1}{n-k+1}}{\binom{n}{k}}$
            \item 
                Derangement: a permutation of the elements of a set, such that no element appears in
                    its
                    original position. Let $d(n)$ be the number of derangements of the identity permutation fo size $n$.
                

                \[d(n)=(n-1) \cdot (d(n-1)+d(n-2)) \, \text{where}\;d(0)=1,d(1)=0\]
            
            \item \textbf{Involutions:} permutations such that $\displaystyle p^2 =$ identity permutation.
                $\displaystyle
                a_0 = a_1 = 1$ and $\displaystyle a_n=a_{n-1} + (n-1)a_{n-2}$ for $n>1$.
            \item 
                Let $T(n,k)$ be the number of permutations of size $n$ for which all cycles have length
                    $\displaystyle
                    \leq
                    k$.

                \(T(n, k)=
                    \begin{cases}
                    n! & ; n \le k\\
                    n \cdot T(n-1, k) - F(n-1, k) \cdot T(n-k-1, k) & ;n > k \\
                    \end{cases}\)
                    Here $\displaystyle F(n, k) = n \cdot (n - 1) \cdot … \cdot (n - k + 1)$

            
            \item \textbf{Lucas Theorem}
                \begin{enumerate}

                    \item If $\displaystyle p$ is prime, then $\displaystyle \left(\frac{p^{a}}{k}\right) \equiv 0 (\mod\;
                        p)$
                    
                    \item 
                        For non-negative integers $m$ and $n$ and a prime $\displaystyle p$, the following congruence
                            relation holds:

                        \(\displaystyle \left(\frac{m}{n}\right) \equiv \prod_{i=0}^{k}
                            \left(\frac{m_{i}}{n_{i}}\right)
                            (mod\; p),\)
                            where,
                            $m=m_{k} p^{k} +m_{k-1} p^{k-1}+…+m_{1} p+m_{0}$,
                            and 
                            $n=n_{k}p^{k}+n_{k-1}p^{k-1}+…+n_{1}p+n_{0}$
                            are the base $\displaystyle p$ expansions of $m$ and $n$ respectively. This uses the
                            convention
                            that
                            $\displaystyle \left(\frac{m}{n}\right)=0$,when $m<n$.

                    
                \end{enumerate}

            
            \item 
                $\displaystyle \sum_{i=0}^{n} {n \choose i } \cdot i^{k}$
                    = $\displaystyle \sum_{i=0}^{n} {n \choose i } \cdot \sum_{j=0}^{k}{ \left\{ \begin{matrix} k\cr j
                    \end{matrix} \right\} \cdot i^{\underline{j}}}$
                    = $\displaystyle \sum_{i=0}^{n} {n \choose i } \cdot \sum_{j=0}^{k} \left\{ \begin{matrix} k\cr j
                    \end{matrix} \right\} \cdot j! {n \choose i }$
                    = $\displaystyle \sum_{i=0}^{n} \frac{n!}{(n-i)!} \cdot \sum_{j=0}^{k} \left\{ \begin{matrix}k\cr j
                    \end{matrix} \right\} \cdot \frac{1}{(i-j)!}$
                    = $\displaystyle \sum_{i=0}^{n} \sum_{j=0}^{k} \frac{n!}{(n-i)!} \cdot \left\{\begin{matrix} k\cr j
                    \end{matrix} \right\} \cdot \frac{1}{(i-j)!}$
                    = n!$\displaystyle \sum_{i=0}^{n} \sum_{j=0}^{k} \left\{ \begin{matrix} k\cr j \end{matrix} \right\}
                    \cdot
                    \frac{1}{(n-i)!} \cdot \frac{1}{(i-j)!}$
                    = n!$\displaystyle \sum_{i=0}^{n} \sum_{j=0}^{k} \left\{ \begin{matrix} k\cr j \end{matrix} \right\}
                    \cdot {
                    n-j \choose n-i} \cdot \frac{1}{(n-j)!}$
                    = n!$\displaystyle \sum_{j=0}^{k} \left\{ \begin{matrix} k\cr j \end{matrix} \right\} \cdot
                    \frac{1}{(n-j)!}
                    \sum_{i=0}^{n} \cdot { n-j \choose n-i}$
                    = $\displaystyle \sum_{j=0}^{k} \left\{\begin{matrix} k\cr j \end{matrix} \right\} \cdot
                    n^{\underline{j}}
                    \cdot 2^{n-j}$

                Here $n^{\underline{j}}=P(n,j)=\dfrac{n!}{(n-j)!}$ and $\displaystyle \left\{ \begin{matrix} k\cr j
                    \end{matrix} \right\}$ is stirling number of the second kind.

                So, instead of $O(n)$, now you can calculate the original equation in $O(k^2)$ or even in $O(k \log^2
                    n)$
                    using NTT.

            
            \item $\displaystyle \displaystyle \sum \limits_{i=0}^{n-1} {i \choose j} x^i = x^j(1-x)^{-j-1}\left( 1- x^n
                \sum
                \limits_{i=0}^j {n \choose i} x^{j-i} (1-x)^i \right)$
            \item 
                $\displaystyle x_{0}, x_{1}, x_{2}, x_{3}, … , x_{n}$
                    $\displaystyle x_{0} + x_{1}, x_{1} + x_{2}, x_{2}+x_{3}, … x_{n}$
                    …
                    If we continuously do this $n$ times then the polynomial of the first column of the $n$-th row will
                    be
                

                \[\displaystyle p(n)=\sum_{k=0}^{n}{n \choose k} \cdot x(k)\]
            
            \item 
                If $\displaystyle P(n)=\sum_{k=0}^{n}{n \choose k} \cdot Q(k)$,
                    then,

                \[Q(n)=\sum_{k=0}^{n}(-1)^{n-k}{n \choose k} \cdot P(k)\]
            
            \item 
                If $\displaystyle P(n)=\sum_{k=0}^{n}(-1)^{k}{n \choose k} \cdot Q(k)$ ,
                    then,

                \[Q(n)=\sum_{k=0}^{n}(-1)^{k}{n \choose k} \cdot P(k)\]
                \end{enumerate}
\subsubsection{Catalan Numbers}\begin{enumerate}

            
            \item $\displaystyle C_n=\frac{1}{n+1}{2n \choose n}$
            \item $\displaystyle C_0=1,C_1=1\text{ and }C_n=\sum \limits_{k=0}^{n-1}C_k C_{n-1-k}$
            \item Number of correct bracket sequence consisting of $n$ opening and $n$ closing brackets.
            \item The number of ways to completely parenthesize $n$+$\displaystyle 1$ factors.
            \item The number of triangulations of a convex polygon with $n$+$\displaystyle 2$ sides (i.e. the number of
                partitions
                of polygon into disjoint triangles by using the diagonals).
            \item The number of ways to connect the $\displaystyle 2n$ points on a circle to form $n$ disjoint i.e.
                non-intersecting chords.
            \item The number of monotonic lattice paths from point $\displaystyle (0,0)$ to point $\displaystyle (n,n)$ in
                a
                square lattice of size $n\times n$, which do not pass above the main diagonal (i.e. connecting
                $\displaystyle
                (0,0)$ to $\displaystyle (n,n)$).
            \item The number of rooted full binary trees with $n$+$\displaystyle 1$ leaves (vertices are not numbered). A
                rooted
                binary tree is full if every vertex has either two children or no children.
            \item Number of permutations of $\displaystyle {1, …, n}$ that avoid the pattern $\displaystyle 123$ (or any
                of
                the
                other patterns of length $3$); that is, the number of permutations with no three-term increasing
                sub-sequence.
                For $n = 3$, these permutations are $\displaystyle 132,\ 213,\ 231,\ 312$ and $321.$\displaystyle For $n
                =
                4$,
                they are $\displaystyle 1432,\ 2143,\ 2413,\ 2431,\ 3142,\ 3214,\ 3241,\ 3412,\ 3421,\ 4132,\ 4213,\
                4231,\
                4312$ and $4321.$
            \item 
                Balanced Parentheses count with prefix:
                    The count of balanced parentheses sequences consisting of $n+k$ pairs of parentheses where the first
                    $\displaystyle \displaystyle k$ symbols are open brackets. Let the number be $\displaystyle
                    C_n^{(k)}$,
                    then
                

                \[\displaystyle \displaystyle C_n^{(k)} = \frac{k+1}{n+k+1} \binom{2n+k}{n}\]
                \end{enumerate}
\subsubsection{Narayana numbers}\begin{enumerate}

            
            \item $N(n,k)=\frac{1}{n}\left(\frac{n}{k}\right)\left(\frac{n}{k-1}\right)$
            \item 
                The number of expressions containing $n$ pairs of parentheses, which are correctly matched and which
                    contain
                    $\displaystyle \displaystyle k$ distinct nestings. For instance, $N(4, 2) = 6$ as with four pairs of
                    parentheses six sequences can be created which each contain two times the sub-pattern ‘()’.

                \end{enumerate}
\subsubsection{Stirling numbers of the first kind}\begin{enumerate}

            
            \item The Stirling numbers of the first kind count permutations according to their number of cycles (counting
                fixed
                points as cycles of length one).
            \item $S(n,k)$ counts the number of permutations of $n$ elements with $\displaystyle \displaystyle k$ disjoint
                cycles.
            
            \item $S(n,k)=(n-1) \cdot S(n-1,k)+S(n-1,k-1),$
                \(where,\; S(0,0)=1,S(n,0)=S(0,n)=0\)
            \item $\displaystyle \displaystyle\sum_{k=0}^{n}S(n,k) = n!$
            \item 
                The unsigned Stirling numbers may also be defined algebraically, as the coefficient of the rising
                    factorial:
                

                \[\displaystyle x^{\bar{n}} = x(x+1)...(x+n-1) = \sum_{k=0}^{n}{ S(n, k) x^k}\]
            
            \item 
                Lets $[n, k]$ be the stirling number of the first kind, then

                \[\displaystyle \bigl[\!\begin{smallmatrix} n \\ n\ -\ k \end{smallmatrix}\!\bigr] = \sum_{0 \leq i_1
                <
                i_2
                < i_k < n}{i_1i_2....i_k.}\]
                \end{enumerate}
\subsubsection{Stirling numbers of the second kind}\begin{enumerate}

            
            \item Stirling number of the second kind is the number of ways to partition a set of n objects into k
                non-empty
                subsets.
            \item $S(n,k)=k \cdot S(n-1,k)+S(n-1,k-1)$,
                \(where \; S(0,0)=1,S(n,0)=S(0,n)=0\)
            \item $S(n,2)=2^{n-1}-1$
            \item $S(n,k) \cdot k!$ = number of ways to color $n$ nodes using colors from $\displaystyle 1$ to
                $\displaystyle
                \displaystyle k$ such that each color is used at least once.
            \item An $r$-associated Stirling number of the second kind is the number of ways to partition a set of $n$
                objects
                into $\displaystyle \displaystyle k$ subsets, with each subset containing at least $r$ elements. It is
                denoted
                by $S_r( n , k )$ and obeys the recurrence relation.
                $\displaystyle \displaystyle S_r(n+1,k) = k S_r(n,k) + \binom{n}{r-1} S_r(n-r+1,k-1)$
            \item 
                Denote the n objects to partition by the integers $\displaystyle 1, 2, …., n$. Define the reduced
                    Stirling
                    numbers of the second kind, denoted $S^d(n, k)$, to be the number of ways to partition the integers
                    $\displaystyle 1, 2, …., n$ into k nonempty subsets such that all elements in each subset have
                    pairwise
                    distance at least d.
                    That is, for any integers i and j in a given subset, it is required that $|i - j| \geq d$. It has
                    been
                    shown
                    that these numbers satisfy,
                    \(S^d(n, k) = S(n - d + 1, k - d + 1), n \geq k \geq d\)

                \end{enumerate}
\subsubsection{Bell number}\begin{enumerate}

            
            \item Counts the number of partitions of a set.
            \item $B_{n+1}=\displaystyle\sum_{k=0}^{n}\left(\frac{n}{k}\right) \cdot B_{k}$
            \item 
                $B_{n}=\displaystyle\sum_{k=0}^{n}S(n,k)$ ,where $S(n,k)$ is stirling number of second kind.

            
        \end{enumerate}
